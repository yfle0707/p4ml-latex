\section{Implementation}
We implemented the \system as a plugin of the BytePs, 
which integrated the PyTorch, TensorFlow, and MxNet.
Based on BytePs, \system intercepts the {\it Push } and {\it Pull} function calls at the workers
and workers communicate with a \system parameter server.
The end host component is written by RAW ETHERNET~\cite{}, 
a Mellanox userspace networking programming model.

\system leverages the hardware acceleration features, i.e., TSO and Multi-Packet QP.
TSO feature can speed up the packet sending rate 
by offloading packetization to the NIC and full utilize the PCIe bandwidth by DMA large chunk of data.
To enhance the packet receiving rate, Multi-Packet QP feature can reduce the NIC memory footprint of at least $512$x smaller receive queue entries. Both features reduce CPU
cost from fewer calls of packet sending and receiving, and fewer PCIe DMA reads for
the NIC to fetch send queue and receive queue entries.
 
 
Switch implementation is challenging due to the two main limitations: 
(1) the switch only has 12 stages and each stage can maximumly process $4$ register value; 
(2) the same register can be only 
accessed once for a packet.
To overcome the stage limitation, we first decide to use $8$ stages
to guarantee to process $128$ bytes data for each packet such that 
our performance can match with \switchml.  
All the control information processing has to be compressed into the remaining $4$ stages. 
The main control information for each aggregator to check follows the following sequence:
(1) isACK; (2)application id; (3)packet sequence number; (4)isResend; (5)bitmap.
As we demonstrated in the seciton~\ref{design}, a packet can do aggregation 
on an aggregator only if the application id and the packet sequence number of 
the packet match with those of the application.
It indicates we can break the dependency from the packet sequence number to the application id.
Thus, we merge the check of application id and packet sequence number in
the same stage such that all the control information processing can be fit into $4$ stages.
To avoid the same register accessed twice in the pipeline, 
\system cleans the control information that is maintained by the register 
in the ack backward path. 
